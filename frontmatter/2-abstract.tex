\chapter*{ABSTRACT\\ \ttitle}
\pagenumbering{roman}
\setcounter{page}{4}

The abstract should consist of a brief, comprehensive summary of the contents of
the thesis. The aim is to allow readers to survey the contents of the thesis
quickly. It should mention the aim of your research, what you did and how you
did it, and the results. It should also indicate the importance of the thesis—what
makes it worth reading, or what it contributes to your field of study. Abstract
length for the Boğaziçi University Institute for Graduate Studies in the Social
Sciences is 250 words maximum, so the abstract should fit onto a single page.
The name of the author does not appear on the abstract page. The Turkish
version of the abstract (with the heading Özet) should reflect the content and
approximate length of the English abstract. As shown above, the word “Abstract”
is capitalized and centered above the title of the thesis. The text of the abstract
itself is double-spaced. Note that the first line is not indented, but begins flush
with the left margin. Normally, an abstract should be a single paragraph. If a
second paragraph is essential, please indent the second paragraph, maintaining
double spacing throughout. 


\newpage

\chapter*{ÖZET\\ \ttitletr}

Bütün insanlar hür, haysiyet ve haklar bakımından eşit doğarlar. Akıl ve vicdana sahiptirler ve birbirlerine karşı kardeşlik zihniyeti ile hareket etmelidirler. Herkes, ırk, renk, cinsiyet, dil, din, siyasi veya diğer herhangi bir akide, milli veya içtimai menşe, servet, doğuş veya herhangi diğer bir fark gözetilmeksizin işbu Beyannamede ilan olunan tekmil haklardan ve bütün hürriyetlerden istifade edebilir. Bundan başka, bağımsız memleket uyruğu olsun, vesayet altında bulunan, gayri muhtar veya sair bir egemenlik kayıtlamasına tabi ülke uyruğu olsun, bir şahıs hakkında, uyruğu bulunduğu memleket veya ülkenin siyasi, hukuki veya milletlerarası statüsü bakımından hiçbir ayrılık gözetilmeyecektir. Yaşamak, hürriyet ve kişi emniyeti her ferdin hakkıdır. Hiç kimse kölelik veya kulluk altında bulundurulamaz; kölelik ve köle ticareti her türlü şekliyle yasaktır. Hiç kimse işkenceye, zalimane, gayriinsani, haysiyet kırıcı cezalara veya muamelelere tabi tutulamaz. Herkes her nerede olursa olsun hukuk kişiliğinin tanınması hakkını haizdir. Kanun önünde herkes eşittir ve farksız olarak kanunun eşit korumasından istifade hakkını haizdir. Herkesin işbu Beyannameye aykırı her türlü ayırdedici muameleye karşı ve böyle bir ayırdedici muamele için yapılacak her türlü kışkırtmaya karşı eşit korunma hakkı vardır \citep{assembly1948universal}. 
