\section{Linguistic trees}

In linguistics we use `trees' that are representational figures that show hierarchical order of how different levels of structural categories are put together. For the most part of the trees you can use the package `forest'. For example see Figure \ref{fig:linguisticexample} for a representation of (\ref{linguisticexample}).

\begin{figure}[hbt!]
    \centering
    \begin{forest}
        [TP 
            [DP\\\textit{Furkan}_i, name=specT]
            [T' 
                [VoiceP 
                    [\sout{DP}_i, name=specVoi]
                    [Voice' 
                        [VP 
                            [DP\\\textit{ev-i}]
                            [V\\\textit{bul}]]
                        [Voice]]]
                [T\\\textit{-du}]]]
    \draw[semithick, dashed, ->] (specVoi) to[out=north west, in=south] (specT);
    \end{forest}
    \caption{An example syntax tree}
    \label{fig:linguisticexample}
\end{figure}

We can also employ feature matrices with our trees as in Figure \ref{fig:linguisticexample2}.
      
\begin{figure}[hbt!]
    \centering
    \begin{forest}
        [TP 
            [\begin{avm}
            \[
                per & {\Third} \\ 
                num & {\Sg} \\
                case & {\Nom} \]
            \end{avm}, name=specT]
            [T' 
                [VoiceP 
                    [\sout{DP}_i, name=specVoi]
                    [Voice' 
                        [VP 
                            [DP\\\textit{ev-i}]
                            [V\\\textit{bul}]]
                        [Voice]]]
                [T\\\textit{-du}]]]
    \draw[semithick, dashed, ->] (specVoi) to[out=west, in=south] (specT);
    \end{forest}
    \caption{Another example syntax tree}
    \label{fig:linguisticexample2}
\end{figure}

If you happen to have any difficulties with drawing trees consult package documentations for `tikz', `tikz-qtree', `forest', and `avm'. This template lets you use all of them in your trees.

\subsection{Additional trees}

\begin{figure}[hbt!]
    \centering
    \begin{tikzpicture}
    \Tree[.N
            [.N 
                [.\textit{tebrik} ]
                [.\textit{ve} ]
                [.\textit{teşekkür} ] ]
            [ 
                [.\textit{-ler} ]
                [.\textit{-im} ] ]
    ];
    \end{tikzpicture}
    \caption{{\Pl} and {\Poss} forming a complex head in grammatical SA}
    \label{fig:furkan2}
\end{figure}

All human beings are born free and equal in dignity and rights. They are endowed with reason and conscience and should act towards one another in a spirit of brotherhood. Everyone is entitled to all the rights and freedoms set forth in this Declaration, without distinction of any kind, such as race, colour, sex, language, religion, political or other opinion, national or social origin, property, birth or other status. Furthermore, no distinction shall be made on the basis of the political, jurisdictional or international status of the country or territory to which a person belongs, whether it be independent, trust, non-self-governing or under any other limitation of sovereignty. Everyone has the right to life, liberty and security of person. No one shall be held in slavery or servitude; slavery and the slave trade shall be prohibited in all their forms. No one shall be subjected to torture or to cruel, inhuman or degrading treatment or punishment. Everyone has the right to recognition everywhere as a person before the law. All are equal before the law and are entitled without any discrimination to equal protection of the law. All are entitled to equal protection against any discrimination in violation of this Declaration and against any incitement to such discrimination.
\begin{figure}[hbt!]
    \centering
\begin{tikzpicture}
    \Tree[.PossP
            [.PlurP 
                [.NP 
                    [.NP\\tebrik ]
                    [.Conj\\ve ]
                    [.\node(NP2){NP}; ] ]
                [.\node(plr){Plur}; ] ]
            [.\node(PS){Poss}; ]
]
\node[below right= 1em of NP2, draw](co){teşekkür-ler-im};
\draw[thick, ->] (NP2) -- (co);
\draw[thick, ->] (plr) -- (co);
\draw[thick, ->] (PS) -- (co);
\end{tikzpicture}
    \caption{Lexical sharing analysis of {\Pl} and {\Poss} in SA}
    \label{fig:lexicalshare}
\end{figure}

All human beings are born free and equal in dignity and rights. They are endowed with reason and conscience and should act towards one another in a spirit of brotherhood. Everyone is entitled to all the rights and freedoms set forth in this Declaration, without distinction of any kind, such as race, colour, sex, language, religion, political or other opinion, national or social origin, property, birth or other status. Furthermore, no distinction shall be made on the basis of the political, jurisdictional or international status of the country or territory to which a person belongs, whether it be independent, trust, non-self-governing or under any other limitation of sovereignty. Everyone has the right to life, liberty and security of person. No one shall be held in slavery or servitude; slavery and the slave trade shall be prohibited in all their forms. No one shall be subjected to torture or to cruel, inhuman or degrading treatment or punishment. Everyone has the right to recognition everywhere as a person before the law. All are equal before the law and are entitled without any discrimination to equal protection of the law. All are entitled to equal protection against any discrimination in violation of this Declaration and against any incitement to such discrimination.
\begin{figure}[hbt!]
    \centering
    \begin{forest}
        [ConjP, s sep=30mm 
            [Conj' 
                [TP 
                    [DP\\\textit{Ahmet}_i]
                    [T'
                        [VoiceP 
                            [\sout{DP}_i]
                            [Voice' 
                                [VP 
                                    [\sout{DP}, name=tk]
                                    [V\\\textit{al}]]
                                [Voice]]]
                        [T\\\textit{-dı}]]]
                [Conj\\\textit{ve}]
                [TP 
                    [DP\\\textit{Mehmet}_j]
                    [T' 
                        [VoiceP 
                            [\sout{DP}_j] 
                            [Voice' 
                                [VP 
                                    [\sout{DP}, name=tl]
                                    [V\\\textit{sat}]]
                                [Voice]]]
                        [T\\\textit{-tı}]]]]
            [DP\\\textit{kitab-ı}, name=DP ]]
        \draw[rounded corners=1em, ->] (tk.south) -- ++(south:2.5em) -| (DP.south);
        \draw[rounded corners=1em, ->] (tl.south) -- ++(south:1.5em) -| (DP.south);
    \end{forest}
    \caption{RNR analysis for Backward Ellipsis}
    \label{fig:backwardellipsis}
\end{figure}

