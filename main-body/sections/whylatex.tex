\section{Why {\LaTeX}?}

Some of the advantages that {\LaTeX} provides include:

\begin{itemize}
    \item Presenting linguistic examples consistently
    \item Representing the data with tailored figures 
    \item Only the references used in-text are listed at the end automatically
    \item Cross-referencing linguistic examples, figures, and tables 
\end{itemize}

\section{Additional tables}
All human beings are born free and equal in dignity and rights. They are endowed with reason and conscience and should act towards one another in a spirit of brotherhood. Everyone is entitled to all the rights and freedoms set forth in this Declaration, without distinction of any kind, such as race, colour, sex, language, religion, political or other opinion, national or social origin, property, birth or other status. Furthermore, no distinction shall be made on the basis of the political, jurisdictional or international status of the country or territory to which a person belongs, whether it be independent, trust, non-self-governing or under any other limitation of sovereignty. Everyone has the right to life, liberty and security of person. No one shall be held in slavery or servitude; slavery and the slave trade shall be prohibited in all their forms. No one shall be subjected to torture or to cruel, inhuman or degrading treatment or punishment. Everyone has the right to recognition everywhere as a person before the law. All are equal before the law and are entitled without any discrimination to equal protection of the law. All are entitled to equal protection against any discrimination in violation of this Declaration and against any incitement to such discrimination.

\begin{table}[hbt!]
\caption{Verbal Terminal Morphemes}
\vspace{10pt}
    \centering
    \begin{tabular}{|ll|}
    \hline 
                                    {(i) Agreement markers} &  \\ \hline
                                    \multirow{5}{20em}{(ii) Aspect/ Modality markers}  & {\Aor} \textit{-(I)r/(A)r} \\ 
                                                            & {\Prog} \textit{-Iyor} \\
                                                            & {\Fut} \textit{-(y)AcAK} \\
                                                            & {\Evi} \textit{-mIş} \\
                                                            & {\Nec} \textit{-mAlI} \\ \hline
        \multirow{2}{20em}{(iii) Converb markers}           & \textit{-(y)IncA} \\
                                                            & \textit{-(y)Ip} \\
    \hline                                                         
    \end{tabular}
    \label{tab:terminalmorphemes}
\begin{flushright}
    Adapted from \cite{kabak2007turkish}
\end{flushright}
\end{table}

All human beings are born free and equal in dignity and rights. They are endowed with reason and conscience and should act towards one another in a spirit of brotherhood. Everyone is entitled to all the rights and freedoms set forth in this Declaration, without distinction of any kind, such as race, colour, sex, language, religion, political or other opinion, national or social origin, property, birth or other status. Furthermore, no distinction shall be made on the basis of the political, jurisdictional or international status of the country or territory to which a person belongs, whether it be independent, trust, non-self-governing or under any other limitation of sovereignty. Everyone has the right to life, liberty and security of person. No one shall be held in slavery or servitude; slavery and the slave trade shall be prohibited in all their forms. No one shall be subjected to torture or to cruel, inhuman or degrading treatment or punishment. Everyone has the right to recognition everywhere as a person before the law. All are equal before the law and are entitled without any discrimination to equal protection of the law. All are entitled to equal protection against any discrimination in violation of this Declaration and against any incitement to such discrimination.
\begin{table}[hbt!]
    \caption{Response of Different Category Suffixes in Korean to Lexical Integrity Tests}
    \vspace{10pt}
    \centering
    \begin{tabular}{|p{1.8cm}|p{2.1cm}|p{1.7cm}|p{1.6cm}|p{1.6cm}|p{2cm}|p{1.8cm}|}
    \hline 
                        & Coordination & External Modifiers & Gapping (Base) & Gapping (Suffix) & Inbound Ana Island & Extraction \\ \hline 
    Opaque Suffix       & N             & N                 & N             & N                 & N   & N \\ \hline 
    Transparent Suffix  & Y             & Y                 & N             & Y                 & Y   & N \\ \hline 
    Double-duty Suffix  & N/Y           & N/Y               & N             & N/Y               & N/Y & N/Y \\ \hline 
    \end{tabular}
    \label{tab:korean}
\end{table}

All human beings are born free and equal in dignity and rights. They are endowed with reason and conscience and should act towards one another in a spirit of brotherhood. Everyone is entitled to all the rights and freedoms set forth in this Declaration, without distinction of any kind, such as race, colour, sex, language, religion, political or other opinion, national or social origin, property, birth or other status. Furthermore, no distinction shall be made on the basis of the political, jurisdictional or international status of the country or territory to which a person belongs, whether it be independent, trust, non-self-governing or under any other limitation of sovereignty. Everyone has the right to life, liberty and security of person. No one shall be held in slavery or servitude; slavery and the slave trade shall be prohibited in all their forms. No one shall be subjected to torture or to cruel, inhuman or degrading treatment or punishment. Everyone has the right to recognition everywhere as a person before the law. All are equal before the law and are entitled without any discrimination to equal protection of the law. All are entitled to equal protection against any discrimination in violation of this Declaration and against any incitement to such discrimination.
\begin{table}[hbt!]
    \caption{Mari Nominal Domain Morpheme Order}
    \vspace{10pt}
    \centering
    \begin{tabular}{|ll|}
    \hline 
        {\Pl} \textgreater {\Poss} & \textit{pasu-vlak-na}  \\
        {\Poss} \textgreater {\Pl} & \textit{pasu-na-vlak} \\ \hline
        {\Pl} \textgreater {\Lcase} & \textit{pasu-vlak-e\u{s}te} \\
        {\Pl} \textgreater {\Scase} & \textit{pasu-vlak-em} \\ \hline
        {\Lcase} \textgreater {\Poss} & \textit{pasu-\u{s}te-na} \\
        {\Poss} \textgreater {\Scase} & \textit{pasu-na-m} \\ \hline
        {\Pl} \textgreater {\Lcase} \textgreater {\Poss} & \textit{pasu-vlak-e\u{s}te-na} \\
        {\Poss} \textgreater {\Pl} \textgreater {\Lcase} & \textit{?pasu-na-vlak-e\u{s}te} \\ \hline
        {\Pl} \textgreater {\Poss} \textgreater {\Scase} & \textit{pasu-vlak-na-m} \\
        {\Poss} \textgreater {\Pl} \textgreater {\Scase} & \textit{pasu-na-vlak-em} \\ \hline 
        \multicolumn{2}{|l|}{\textit{pasu} `garden', \textit{-vlak} {\Pl}, \textit{-na} {\Poss}.{\First}.{\Pl}, \textit{-(e)\u{s}te} {\Iness}, \textit{-(e)m} {\Acc}} \\
        \hline 
    \end{tabular}
    \label{tab:mariorder} \\
    ${}$ \\ \hfill Adapted from \cite{guseva2017postsyntactic}
\end{table}

\begin{table}[hbt!]
    \caption{Feature Geometry of {\Poss} in Turkish}
    \centering
    \begin{tabular}{|l|l|l|}
    \hline
         \multicolumn{2}{|c|}{Features} & \multirow{2}{*}{Exponent}  \\ \cline{1-2}
         Participant & Individuation  & \\ \hline
         Speaker & $\emptyset$ & \textit{-Im} \\ \hline 
         Addressee & $\emptyset$ & \textit{-In} \\ \hline 
         Speaker & Group & \textit{-ImIz} \\ \hline 
         Addressee & Group & \textit{-InIz} \\ \hline 
         $\emptyset$ & $\emptyset$ & \textit{-(s)I(n)} \\ \hline 
         $\emptyset$ & Group & \textit{-lArI} \\ \hline 
    \end{tabular}
    \label{tab:kharyfeatures}
\end{table}